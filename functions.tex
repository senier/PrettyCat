\documentclass[a4paper]{article}
\usepackage[pagebackref=true]{hyperref}
\usepackage[hyperref]{xcolor}
\usepackage{geometry}
\usepackage{amssymb}
\usepackage{mathtools}
\usepackage{tikz}
\usepackage{dot2texi}
\usepackage{algorithm2e}
\usepackage{pdflscape}

\setlength{\parskip}{.5em}
\setlength{\parindent}{0em}

\DeclareMathOperator{\hash}{Hash}
\DeclareMathOperator{\hashvrfy}{Verify_{Hash}}
\DeclareMathOperator{\hmac}{HMAC}
\DeclareMathOperator{\hmacvrfy}{Verify_{HMAC}}
\DeclareMathOperator{\enc}{Encrypt}
\DeclareMathOperator{\dec}{Decrypt}
\DeclareMathOperator{\sign}{Sign}
\DeclareMathOperator{\signvrfy}{Verify_{Sig}}
\DeclareMathOperator{\id}{Id}
\DeclareMathOperator{\trans}{Transform}
\DeclareMathOperator{\rand}{Rand}
\DeclareMathOperator{\dhsec}{DH_{Sec}}
\DeclareMathOperator{\dhpub}{DH_{Pub}}
\DeclareMathOperator{\send}{Send}
\DeclareMathOperator{\recv}{Receive}
\DeclareMathOperator{\guard}{Guard}

\newcommand{\TODO}[1]{\small\noindent\color{red} TODO: #1\color{black}}

\newcommand{\emptysec}{\varnothing}
\newcommand{\secminus}{\smallsetminus}

\definecolor{myblue}{HTML}{0091FF}
\hypersetup
{
    colorlinks = true,
    linkcolor  = myblue,
    citecolor  = myblue
}

\begin{document}

\section{\label{sec:notation}Notation}

Security requirements are denoted by $c$ (Confidentiality), $i$ (Integrity) and
$f$ (Freshness). The overall security requirement some domain $D$ must fulfill
is denoted by a set $S_D$ over these properties. For a domain $D_1$ which has
to guarantee integrity, but no other properties, this would be:

$$S_{D_1} = \{i\}$$

A domain $D_2$ which does not (need) to fulfill any security requirements has
the following set:

$$S_{D_2} = \emptysec$$

Operations that make up a cryptographic protocol impose preconditions on the
requirement set of their input parameters. These preconditions need to be
guaranteed by the calling domain. Likewise, the results of an operation may
impose security requirements on the receiving domain.

Note, that this is opposite to e.g. Hoard logic (citation needed) where the
execution of a program is guaranteed to establish some postcondition, given its
precondition is satisfied. In our logic we rather remove the requirements that
an operation guarantees from the resulting security requirements set only
leaving those that still need to be guaranteed by the receiving domain. This
enables us to reason about domain-wide requirements and to split up domains
with equal into isolated partitions to reduce the trusted computing base.

An operation $O$ has input parameters $i_k$ and output parameters $o_j$ which
each have an associated security requirement set $\iota_k$ and $\omega_j$
respectively:

$$P(i_1^{\iota_1}, \cdots, i_n^{\iota_n}) \mapsto (o_1^{\omega_1}, \cdots, o_m^{\omega_m})$$

How the resulting security requirements sets relate to the sets of the input
parameters depends on the semantics of an operation $O$. The most important
primitives used in cryptographic protocols are detailed in the following
section.

% TODO mention $\sigma$ as placeholder
% TODO examples

\section{Primitives}

\subsection{Messaging}

Send the message $msg$ through a domain guaranteeing the security properties $\sigma$:

$$\send^\sigma(msg^\sigma) \mapsto \emptysec$$

Receive a message $msg$ from a domain guaranteeing the security properties $\sigma$:

$$\recv^\sigma \mapsto msg^\sigma$$

\subsection{Data processing}

$$const^{\sigma} \mapsto const^{\sigma}$$

$$\trans(in_1^{\sigma_1}, \cdots, in_n^{\sigma_n}) \mapsto (out_1^{\cup\sigma_i},\cdots,out_m^{\cup\sigma_i})$$

The guard primitive outputs $data$ unmodified if $cond$ does not equal
$\emptysec$. This can be used to model situations where some data, e.g. key
material, must only be used after the key exchange has been validated.

$$\guard(data^{\sigma}, cond^{\{i\}}) \mapsto data^{\sigma} | \emptysec$$

\subsection{Random Numbers}

$$\rand(len^{\{i\}}) \mapsto data^{\{c,i,f\}}$$

\subsection{Hashing}

A cryptographic hash function efficiently computes a fixed-length hash value
$h$ for any message $m$ of arbitrary length. They are often called one-way hash
functions, as it is infeasible for an attacker to recreate the input message
from it hash value or to find two different messages resulting in the same hash
value. Furthermore, even smallest changes in an input message result in
significant changes in the resulting hash.

As a hash $h$ uniquely identifies an input message $m$, it can be used to prove
knowledge of $m$ without revealing it. A hash function thus eliminates a
potential confidentiality requirement in the input security set, but does not
influence integrity or freshness requirements:

$$\hash(msg^\sigma) \mapsto hash^{\sigma\secminus\{c\}}$$

\TODO{Is freshness really not influenced by a hash function?}

The inverse function\footnote{Obviously, an efficient inverse function to a
one-way hash should not exist. We rather refer to an inverse function as the
operation that is typically used by the receiver of a hash to validate that a
message yields a specific hash value.} takes a message $m$ and a hash
value $h$ as input and returns the boolean value of $hash(m) = h$.

$$\hashvrfy(hash^{i}, msg^\sigma) \mapsto msg^{\sigma} | \emptysec$$

\TODO{How does $\hashvrfy$ influence the security requirement sets?}

\subsection{Message authentication}

$$\hmac(key^{\{c,i\}}, msg^\sigma) \mapsto auth^{\sigma\secminus{\{i\}}}$$

$$\hmacvrfy(key^{\{c,i\}}, auth^\emptysec, msg^\sigma) \mapsto msg^{\sigma\cup\{i\}} | \emptysec$$

% While it looked interesting to never pass input data back in those functions,
% this does not seem to work consistently. If hmacvrfy() only returns a
% verdict on whether the MAC was valid for a message msg and we get pass on the
% message from the original source (i.e. the source hmacvrfy() got msg from,
% then we need to model a case where hmacvrfy() fails and msg does NOT flow
% from source to destination. This looks very ugly.
%
% Alternative idea: Never pass on explicit verdict, but the data or NIL.

\subsection{Digital Signatures}

$$\sign(pkey^{\{i\}}, skey^{\{c,i\}}, msg^\sigma) \mapsto auth^{\sigma\secminus\{i\}}$$

$$\signvrfy(pkey^{\{i\}}, auth^{\emptysec}, msg^\sigma) \mapsto msg^{\sigma\cup\{i\}} | \emptysec$$

\subsection{Diffie-Hellman Key Exchange}

$$\dhpub(gen^{\{i\}}, sec^{\{c,i\}}) \mapsto pub^\emptysec$$

$$\dhsec(pub^\emptysec, sec^{\{c,i\}}) \mapsto sec^{\{c,i\}} $$

\subsection{Symmetric Encryption}

$$\enc((iv^{\{i\}}, key^{\{c,i\}})^{\{f\}}, plaintext^\sigma) \mapsto ciphertext^{\sigma\secminus\{c\}}$$

\TODO{The freshness notation still looks a bit quirky}

$$\dec(iv^{\{i\}}, key^{\{c,i\}}, ciphertext^\sigma) \mapsto plaintext^{\sigma\cup\{c\}}$$

\section{Security Requirements Graph}

Security requirements are not constant during a security protocol run. While
user input to a security messaging protocol must by kept confidential,
symmetric encryption operations remove this requirement such that the
information can be passed to a domain that does not guarantee confidentiality,
like the Internet. To identify different domains implied by a security protocol
and to know what guarantees each part of the needs to provide, a security
requirements graph for that protocol must be created systematically.

As we are interested in splitting up security protocols into components, we do
not associated a security level with the data flowing through the protocol
elements, as done in data flow analysis. Instead, data is tagged with a set of
security requirements (confidentiality, integrity, freshness) the environment
processing this data has to meet. Operations such as the primitives described
in the previous section take tagged data as their input and output tagged data
with potentially different sets of security requirements.

A security requirements graph consists of a set of nodes each of which
represents an operation as defined in \autoref{sec:notation}. Directed edges
between two nodes are labeled with a tuple $(src, dst, sec)$. The element $src$
denotes the output parameter of the sending node and $dst$ the respective input
parameter of the receiving node. The last element, $sec$, is the set of
security requirements the \emph{environment} storing or transferring the data
must guarantee.

\begin{figure}[ht]
    \centering
    \begin{dot2tex}[mathmode]
        digraph G
        {
            #rankdir=LR;
            node[shape=rect];
            rbits [label="128^{\{i\}}"];
            iv [label="0^{\{i\}}"];
            rand [label="Rand"];
            encrypt [label="Encrypt"];
            receive [label="Receive^{\{c\}}"];
            send [label="Send^\emptysec"];

            rbits -> rand [label="(const, len, ?)"];
            rand -> encrypt [label="(data, key, ?)"];
            receive -> encrypt [label="(msg, plaintext, ?)"];
            iv -> encrypt [label="(const, iv, ?)"];
            encrypt -> send [label="(ciphertext, msg, ?)"];
        }
    \end{dot2tex}
    \caption{Security requirements graph for a simplistic encryption scheme}
\end{figure}

\section{Algorithm}

In the initial graph, nodes may still have undetermined security requirements
set. This is true for operations like $\enc$ which take an unspecified set and
perform some transformation on it (like removing the confidentiality
requirement in the case of $\enc$). What is also missing initially, are the
security requirements sets for the edges which are later used to partition the
graph. The following algorithm calculates all undetermined requirements sets
and assigns requirements sets to all edges.

\TODO{Define undetermined}

\begin{algorithm}
    \SetKwData{Start}{start}
    \KwData{Graph $G = \{V, A\}$}
    \KwResult{Graph G with all security sets determined}
    \Start $\leftarrow$ \emph{source nodes of G}\;
    \For{}
    {
    }
\end{algorithm}

\begin{landscape}
\thispagestyle{empty}
    \begin{figure}[ht]
        \centering
        \begin{dot2tex}[mathmode,scale=0.4]
            \input{initiator.dot}
        \end{dot2tex}
        \caption{OTR initiator}
    \end{figure}
\end{landscape}

\end{document}
