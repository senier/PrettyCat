\documentclass[a4paper]{article}
\usepackage[pagebackref=true]{hyperref}
\usepackage[hyperref]{xcolor}
\usepackage[margin=1.5cm]{geometry}
\usepackage{amssymb}
\usepackage{mathtools}
\usepackage{tikz}
\usepackage{dot2texi}

\setlength{\parskip}{.5em}
\setlength{\parindent}{0em}

\newcommand{\TODO}[1]{\small\noindent\color{red} TODO: #1\color{black}}

\begin{document}

\section{Primitives}

Single message security guarantees:

\begin{itemize}
    \item [C] Confidentiality
    \item [I] Integrity
\end{itemize}

Channel security guarantees:

\begin{itemize}
    \item [O] Order-preservation
    \item [NI] No insertion
    \item [NR] No removal
    \item [HMK] Sensitive to soft ice cream
\end{itemize}

A guarantee for interface $arg$ is defined as:

$arg^{\{C|I|O|NI|NR\}}_{\{in|out\}}$

\TODO{Thorsten notes that this is similar to the capabilities of a Dolev-Yao attacker: Relay/Suppress/Save, Manipulate, Forge, (Delay)/Inject. We should at least think about the implication of a channel having unwanted storage capabilities (Example: Collecting all your sensitive emails and flushing them to an unintended receiver at once)!}

\subsection{Data transformation}

\begin{tabular}{p{.2\linewidth}|p{.8\linewidth}}
    \begin{dot2tex}[mathmode]
        digraph G
        {
            node[shape=none, label=""];
            element[label="const", shape="rect"];
            element -> sink [taillabel="const"];
        }
    \end{dot2tex}
    & \\
\end{tabular}

\begin{tabular}{p{.2\linewidth}|p{.8\linewidth}}
    \begin{dot2tex}[mathmode]
        digraph G
        {
            node[shape=none, label=""];
            element[label="xform", shape="rect"];
            source -> element [taillabel="in_i"];
            element -> sink [taillabel="out_i"];
        }
    \end{dot2tex}
    & \\
\end{tabular}

\subsection{Comparators}

\begin{tabular}{p{.2\linewidth}|p{.8\linewidth}}
    \begin{dot2tex}[mathmode]
        digraph G
        {
            node[shape=none, label=""];
            element[label="comp", shape="rect"];
            arg1 -> element [headlabel="left"];
            arg2 -> element [headlabel="right"];
            element -> result [taillabel="result"];
        }
    \end{dot2tex}
    & \\
\end{tabular}

\begin{tabular}{p{.2\linewidth}|p{.8\linewidth}}
    \begin{dot2tex}[mathmode]
        digraph G
        {
            node[shape=none, label=""];
            element[label="guard", shape="rect"];
            data_i -> element [headlabel="data"];
            cond -> element [headlabel="cond"];
            element -> data_o [taillabel="data"];
        }
    \end{dot2tex}
    & \\
\end{tabular}

\begin{tabular}{p{.2\linewidth}|p{.8\linewidth}}
    \begin{dot2tex}[mathmode]
        digraph G
        {
            node[shape=none, label=""];
            element[label="permute", shape="rect"];
            in -> element [taillabel="in_i"];
            order -> element [taillabel="order"];
            element -> out [taillabel="out_i"];
        }
    \end{dot2tex}
    & \\
\end{tabular}

\subsection{Declassification}

\begin{tabular}{p{.2\linewidth}|p{.8\linewidth}}
    \begin{dot2tex}[mathmode]
        digraph G
        {
            node[shape=none, label=""];
            element[label="release", shape="rect"];
            data_i -> element [headlabel="data"];
            element -> data_o [taillabel="data"];
        }
    \end{dot2tex}
    & \\
\end{tabular}

\subsection{Random number and freshness}

\begin{tabular}{p{.2\linewidth}|p{.8\linewidth}}
    \begin{dot2tex}[mathmode]
        digraph G
        {
            node[shape=none, label=""];
            element[label="rng", shape="rect"];
            source -> element [headlabel="len"];
            element -> sink [taillabel="rand"];
        }
    \end{dot2tex}
    & \\
\end{tabular}

\begin{tabular}{p{.2\linewidth}|p{.8\linewidth}}
    \begin{dot2tex}[mathmode]
        digraph G
        {
            node[shape=none, label=""];
            element[label="counter", shape="rect"];
            init -> element [headlabel="init"];
            key_i  -> element [headlabel="key"];
            element -> ctr [taillabel="ctr"];
            element -> key_o [taillabel="key"];
        }
    \end{dot2tex}
    & \\
\end{tabular}

\subsection{Diffie-Hellman}

\begin{tabular}{p{.2\linewidth}|p{.8\linewidth}}
    \begin{dot2tex}[mathmode]
        digraph G
        {
            node[shape=none, label=""];
            element[label="dhpub", shape="rect"];
            gen -> element [headlabel="gen"];
            psec_i -> element [headlabel="psec"];
            element -> pub [taillabel="pub"];
            element -> psec_o [taillabel="psec"];
        }
    \end{dot2tex}
    & \\
\end{tabular}

\begin{tabular}{p{.2\linewidth}|p{.8\linewidth}}
    \begin{dot2tex}[mathmode]
        digraph G
        {
            node[shape=none, label=""];
            element[label="dhsec", shape="rect"];
            pub -> element [headlabel="pub"];
            psec -> element [headlabel="psec"];
            element -> ssec [taillabel="ssec"];
        }
    \end{dot2tex}
    & \\
\end{tabular}

\subsection{Cryptographic Hashes}

\begin{tabular}{p{.2\linewidth}|p{.8\linewidth}}
    \begin{dot2tex}[mathmode]
        digraph G
        {
            node[shape=none, label=""];
            element[label="hash", shape="rect"];
            msg -> element [headlabel="msg"];
            element -> hash [taillabel="hash"];
        }
    \end{dot2tex}
    & \\
\end{tabular}

\subsection{Symmetric encryption}

\begin{tabular}{p{.3\linewidth}|p{.7\linewidth}}
    \begin{dot2tex}[mathmode]
        digraph G
        {
            node[shape=none, label=""];
            element[label="enc_{ctr}", shape="rect"];
            pt  -> element [headlabel="pt"];
            key -> element [headlabel="key"];
            ctr -> element [headlabel="ctr"];
            element -> ct [taillabel="ct"];
        }
    \end{dot2tex}
    & \\
\end{tabular}

\begin{tabular}{p{.3\linewidth}|p{.7\linewidth}}
    \begin{dot2tex}[mathmode]
        digraph G
        {
            node[shape=none, label=""];
            element[label="dec_{ctr}", shape="rect"];
            ct  -> element [headlabel="ct"];
            key -> element [headlabel="key"];
            ctr -> element [headlabel="ctr"];
            element -> pt [taillabel="pt"];
        }
    \end{dot2tex}
    & \\
\end{tabular}

\subsection{Message authentication}

\begin{tabular}{p{.2\linewidth}|p{.8\linewidth}}
    \begin{dot2tex}[mathmode]
        digraph G
        {
            node[shape=none, label=""];
            element[label="hmac", shape="rect"];
            key -> element [headlabel="key"];
            msg_i -> element [headlabel="msg"];
            element -> msg_o [taillabel="msg"];
            element -> auth [taillabel="auth"];
        }
    \end{dot2tex}
    & \\
\end{tabular}

\subsection{Digital Signatures}

\begin{tabular}{p{.3\linewidth}|p{.7\linewidth}}
    \begin{dot2tex}[mathmode]
        digraph G
        {
            node[shape=none, label=""];
            element[label="sign", shape="rect"];
            msg_i -> element [headlabel="msg"];
            pkey -> element [headlabel="pkey"];
            skey -> element [headlabel="skey"];
            element -> msg_o [taillabel="msg"];
            element -> sig [taillabel="sig"];
        }
    \end{dot2tex}
    & \\
\end{tabular}

\begin{tabular}{p{.3\linewidth}|p{.7\linewidth}}
    \begin{dot2tex}[mathmode]
        digraph G
        {
            node[shape=none, label=""];
            element[label="vrfy\_sig", shape="rect"];
            msg_i -> element [headlabel="msg"];
            pkey -> element [headlabel="pkey"];
            sig -> element [headlabel="pkey"];
            element -> msg_o [taillabel="msg"];
        }
    \end{dot2tex}
    & \\
\end{tabular}

\end{document}
