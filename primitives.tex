\documentclass[a4paper,twocolumn]{article}
\usepackage[pagebackref=true]{hyperref}
\usepackage[hyperref]{xcolor}
\usepackage[margin=1.5cm]{geometry}
\usepackage{amssymb}
\usepackage{mathtools}
\usepackage{tikz}
\usepackage{dot2texi}
\usepackage{floatflt}

\setlength{\parskip}{.5em}
\setlength{\parindent}{0em}

\newcommand{\TODO}[1]{\small\noindent\color{red} TODO: #1\color{black}}
\newcommand{\cm}{\checkmark}

%FIXME: Remove unused operators
\DeclareMathOperator{\hash}{Hash}
\DeclareMathOperator{\hmac}{HMAC}
\DeclareMathOperator{\encctr}{Enc_{ctr}}
\DeclareMathOperator{\decctr}{Dec_{ctr}}
\DeclareMathOperator{\sign}{Sign}
\DeclareMathOperator{\signvrfy}{Verify_{Sig}}
\DeclareMathOperator{\rng}{RNG}
\DeclareMathOperator{\dhsec}{DH_{Sec}}
\DeclareMathOperator{\dhpub}{DH_{Pub}}
\DeclareMathOperator{\guard}{Guard}
\DeclareMathOperator{\release}{Release}
\DeclareMathOperator{\const}{Const}
\DeclareMathOperator{\permute}{Permute}
\DeclareMathOperator{\transform}{Transform}
\DeclareMathOperator{\comp}{Comp}
\DeclareMathOperator{\streamcomp}{Scomp}
\DeclareMathOperator{\counter}{Counter}

\newcommand{\geno}{GEN\mbox{-}O{}}
\newcommand{\genm}{GEN\mbox{-}M{}}
\newcommand{\genc}{GEN\mbox{-}C{}}

\begin{document}

\section{Primitives}

This section describes primitive operation a protocol is composed of. Right
now, only the operation necessary to specify (parts of?) OTR are covered. A
primitive (or component) can be treated as a single-threaded process executing
independently inside a concurrent system, i.e. there is no concurrency inside a
primitive, but primitive can execute concurrently.

\subsection{Background}

Every primitive has a set of incoming interfaces denoted as $if_{in}$ and a
set of outgoing interfaces $if_{out}$. Interfaces between components are
connected by channels. For its incoming and outgoing interfaces a primitive may
require security guarantees from the environment by which the respective
channel is controlled. The environment may guarantee more security properties
than required by a primitive.

It is important to note that these guarantees neither express security
properties established by a primitive (like confidentiality through encryption)
nor specific properties that are to be established by the environment (like
freshness). Instead, the environment is trusted to maintain particular
guarantees required by the primitives. A decryption primitive may, for example,
require its environment to maintain confidentiality for its outgoing plaintext
interface, as after decryption confidentiality is not maintained by
cryptography anymore.

This notion differs significantly from techniques like dynamic taint analysis
which are based on the flow of data through a program \cite{5504796}.  In taint
analysis, confidential data is labeled with a single property (the taint) and
the propagation of taints is checked throughout program execution according to
a taint policy. While this is useful to detect misuse of input data, analyze
malware or generate test cases, we have a different goal that data flow
analysis cannot achieve.

The reason for our somewhat counter-intuitive notion of security guarantees the
environment has to meet stems from our goal to automatically derive an ideal
compartmentalization of a security protocol from its specification. Of cause,
for the resulting decomposed protocol the security guarantees of the original
protocol need have to hold. To gain security and reliability when implementing
a protocol, an optimal decomposition minimizes the number and size of trusted
portions while moving as much functionality into isolated less trusted
components of the system \cite{10.1109/COLCOM.2005.1651218}. 

\subsection{Notation}

Single message security guarantees:

\begin{itemize}
    \item [C] Confidentiality
    \item [I] Integrity
\end{itemize}

Channel security guarantees:

\begin{itemize}
    \item [O] Order-preservation
    \item [NI] No insertion
    \item [NR] No removal
\end{itemize}

A guarantee for interface $hmk$ is defined as:

$hmk^{\{C|I|O|NI|NR\}}_{\{in|out\}}$

\TODO{To be discussed: $I$ could be removed in favor of $NI \wedge NR$ as these properties enforce per-message integrity. Furthermore, we never saw requirements for $NI$ or $NR$ alone. Hence, I'd suggest we only keep C, I, and M where M stands for meta-data integrity and is equivalent to $NI \wedge NR$.}

\subsection{Open issues}

We do not consider or model timing and hence cannot reason about covert
channels. Given the associated complexity we want to keep time out of our model
if possible

It may be worthwhile considering meta-data entropy (e.g. the entropy of
data fed into a hash primitive to reason about confidentiality of the output
data)

We talk about Integrity and Meta-Integrity (i.e. Ordering, No insertion, No
removal). It looks like Integrity is a subset of Meta-Integrity, as changing a
message could be modelled as removing a message and inserting a custom one.
Hence, by either achieving 'No removal' or 'No insertion' we could achieve
message integrity. May be helpful to get rid of the redundant per-message
integrity for the sake of reasoning about algorithms.

Should we model something like Meta-Confidentiality? That could cover e.g. the
length of packets.

\begin{figure}[ht]
    \centering
    \begin{dot2tex}[mathmode]
        digraph G
        {
            rankdir=LR
            node[shape=none, label=""];
            source[label="src", shape="rect"];
            sink[label="sink", shape="rect"];
            source -> sink [headlabel="in", taillabel="out"];
        }
    \end{dot2tex}
\end{figure}

What is the relation of input and output interface between a source and a sink?
I used to model this by equivalence, but it feels like implication may be more
suitable to allow for situations where data is sent from a primitive with
weaker requirements to one with stronger requirements:

\begin{equation}
    sink_{in}^{C} \implies src_{out}^{C}
\end{equation}
\begin{equation}
    sink_{in}^{M} \implies src_{out}^{M}
\end{equation}
\begin{equation}
    sink_{in}^{O} \implies src_{out}^{O}
\end{equation}

\subsection{Generic rules}

\TODO{Tell about the overall approach (system of boolean equations, use SMT solver to find solution, later optimize) and what those those rules play.}

When analyzing the rules imposed by primitives on their input and output
interfaces, we identified two generic rules for confidentiality and integrity
that are applicable to multiple (but not all) primitives.

In a primitive without special cryptographic properties, confidentiality
requirements of any input interfaces disseminate to all output interfaces. The
reason behind that is, that if an input interface may indeed carry confidential
data if its environment has the respective guarantee. As nothing is known in
general about how inputs are related to outputs, the only save assumption is
that confidential data taints all outputs. This property is expressed by the
generic confidentiality rule \genc{}:

\begin{equation}
    \bigvee_{i\in I}if_{in_i}^{C} \implies \bigwedge_{o\in O}if_{out_o}^{C}
\end{equation}

By contrast, integrity spreads in the opposite direction, i.e. from outputs to
inputs. The intuitive rationale is that if a generic primitive wants to
guarantee integrity on its output interfaces, all of the input interfaces, too,
must guarantee integrity as any single input could otherwise invalidate
integrity of all outputs. This is expressed in the generic integrity rule
\genm{}:

\begin{equation}
    \bigvee_{o\in O}if_{out_o}^{M} \implies \bigwedge_{i\in I}if_{in_i}^{M}
\end{equation}

The generic order-preservation rule \geno{} is similar to the generic integrity
rule. If any output parameter requires order preservation, the all input
parameter have to require it in the general case:

\begin{equation}
    \bigvee_{o\in O}if_{out_o}^{O} \implies \bigwedge_{i\in I}if_{in_i}^{O}
\end{equation}

Note, that the generic rules necessarily are an over approximation, as we
calculate the security guarantees of the input interfaces from the output
interfaces of a component (and vice versa). However, just because the
environment is required to guarantee e.g. confidentiality does not imply that
confidential data is ever transmitted there. Hence, if we calculate the output
interfaces based on these maximum guarantees, the output guarantees we require
may also be too strong.

\subsection{Data transformation}

\subsubsection{Constant values}

The $\const$ primitive is used to model system parameters like key lengths,
protocol versions or private keys. It always outputs a constant value on its
$const$ output interface.

\begin{figure}[ht]
    \centering
    \begin{dot2tex}[mathmode]
        digraph G
        {
            rankdir=LR;
            node[shape=none, label=""];
            element[label="\const", shape="rect"];
            element -> sink [taillabel="const"];
        }
    \end{dot2tex}
    \caption{$\const$}
\end{figure}

\TODO{The const primitive feels a bit quirky still, as we say we model our system
as concurrent single-threaded components with message passing (while ignoring
time, duh). So, when does const send a message then? Pull-based? By magic?
Regularly?}

Security guarantees:

The generic \genm{} rule holds.

\subsubsection{Transformation of data}

\begin{figure}[ht]
    \centering
    \begin{dot2tex}[mathmode]
        digraph G
        {
            rankdir=LR;
            node[shape=none, label=""];
            element[label="\transform", shape="rect"];
            source -> element [taillabel="in_i"];
            element -> sink [taillabel="out_i"];
        }
    \end{dot2tex}
    \caption{$\transform$}
\end{figure}

The $\transform$ component is a generic primitive to model arbitrary data
transformation without any special influence on the security of the processed
data.  This primitive can be used to prepend headers, to split data or to
assemble whole messages.

Security guarantees:

The generic confidentiality rule \genc{} and the generic integrity rule \genm{} apply.

\subsection{Comparators}

\subsubsection{Equality of data}

\begin{figure}[ht]
    \centering
    \begin{dot2tex}[mathmode]
        digraph G
        {
            rankdir=LR;
            node[shape=none, label=""];
            element[label="\comp", shape="rect"];
            arg1 -> element [headlabel="data_1"];
            arg2 -> element [headlabel="data_2"];
            element -> result [taillabel="result"];
        }
    \end{dot2tex}
    \caption{$\comp$}
\end{figure}

The secure comparator takes two arbitrary inputs, compares them and outputs a
boolean value, depending on whether both values were identical. To prevent
leakage of confidential data, that primitive is assumed to prevent an
extraction of arbitrary bits of the input message.

Security guarantees:

The generic generic integrity rule \genm{} and the generic order-preservation rule \geno{} apply.

\TODO{Can't reconstruct this anymore. If the result does not require order preservation, then by \geno{} the inputs are not required to preserve the order either. Hence, $data_1$ and $data_2$ may be reordered. But then unrelated values are and the $result$ is not only out of order,
but plain wrong. How do we treat this?}

\subsubsection{Equality of data stream elements}

\begin{figure}[ht]
    \centering
    \begin{dot2tex}[mathmode]
        digraph G
        {
            rankdir=LR;
            node[shape=none, label=""];
            element[label="\streamcomp", shape="rect"];
            data -> element [headlabel="data"];
            element -> result [taillabel="result"];
        }
    \end{dot2tex}
    \caption{$\streamcomp$}
\end{figure}

The secure stream comparator $\streamcomp$ takes a single input on the $data$
interface and compares it with the previous message received on the same
interface. Depending on whether the current value equals the previous value a
boolean result is emitted on the outgoing $result$ interfaces. This primitive
can be used conjunction with the $\counter$ primitive to detect when a key did
not change and a counter needs to be incremented to establish freshness.

Security guarantees:

The generic generic integrity rule \genm{} and the generic
order-preservation rule \geno{} apply.

\subsubsection{Data flow control}

\begin{figure}[ht]
    \centering
    \begin{dot2tex}[mathmode]
        digraph G
        {
            rankdir=LR
            node[shape=none, label=""];
            element[label="\guard", shape="rect"];
            data_i -> element [headlabel="data"];
            cond -> element [headlabel="cond"];
            element -> data_o [taillabel="data"];
        }
    \end{dot2tex}
    \caption{$\guard$}
\end{figure}

The $\guard$ primitive controls the data flow between the input $data$ and the
output $data$ interface depending on the value received on the $cond$ input
interfaces. It can be used where otherwise unrelated message depend on each
other in a security protocol, e.g. where the protocol requires a message to be
send only \emph{after} the signature of previous message has been checked.

Security guarantees:

As any value received on the $cond$ input interface relates to one particular
message on the input $data$ interfaces, the order of both input interfaces
needs to be preserved:

\begin{equation}
    data_{in}^{O} \wedge cond_{in}^{O}
\end{equation}

Furthermore, the integrity of the input interfaces must be maintained, as
otherwise an attacker could change the condition to an undesired value, insert
own or remove values:

\begin{equation}
    data_{in}^{M} \wedge cond_{in}^{M}
\end{equation}

Apart from that, for all $data$ interfaces the generic rules \genc{}, \geno{} and
\genm{} need to hold.

\subsubsection{Permutation based on untrusted input}

\begin{figure}[ht]
    \centering
    \begin{dot2tex}[mathmode]
        digraph G
        {
            rankdir=LR
            node[shape=none, label=""];
            element[label="\permute", shape="rect"];
            in -> element [taillabel="data_i"];
            order -> element [taillabel="order"];
            element -> out [taillabel="data_i"];
        }
    \end{dot2tex}
    \caption{$\permute$}
\end{figure}

The $\permute$ primitive changes the flow from its input $data$ interfaces to
its output $data$ interfaces based on the $order$ input interfaces. As $order$
has no requirement for any security guarantees, this construct can be used to
model situations where confidential data is processed based on untrusted input.
An example is a key identifier received over an untrusted network (like the key
IDs in OTR or the SPI in IPsec) which is necessary to select the right security
parameters but which should not influence the security properties of the key
material.

Security guarantees:

For the $data$ interfaces \genc{}, \geno{} and \genm{} hold. Due to the nature of the
component, no requirements are imposed on the $order$ input interface.

\subsection{Declassification}

\begin{figure}[ht]
    \centering
    \begin{dot2tex}[mathmode]
        digraph G
        {
            rankdir=LR
            node[shape=none, label=""];
            element[label="\release", shape="rect"];
            data_i -> element [headlabel="data"];
            element -> data_o [taillabel="data"];
        }
    \end{dot2tex}
    \caption{$\release$}
\end{figure}

Some protocols require a way to model an unconditional release of data, i.e.
the release of all security requirement from the receiving environment without
actually changing the data (as done e.g. in signature or encryption
primitives). The decision whether and when to release data is typically made by
another trusted component. For example, the OTR protocol reveals old MAC keys
once they are not in active use anymore to achieve deniability.

Security guarantees:

The generic rule \genm{} holds.

\subsection{Random number and freshness}

\subsubsection{True random numbers}

\begin{figure}[ht]
    \centering
    \begin{dot2tex}[mathmode]
        digraph G
        {
            rankdir=LR
            node[shape=none, label=""];
            element[label="\rng", shape="rect"];
            source -> element [headlabel="len"];
            element -> sink [taillabel="data"];
        }
    \end{dot2tex}
    \caption{$\rng$}
\end{figure}

The true random number generator outputs the number random bits specified on
input interfaces $len$ onto its output interface $data$.

Security guarantees:

The generic rules \genm{} and \geno{} hold.

\subsubsection{Key-dependent monotonic counters}

\begin{figure}[ht]
    \centering
    \begin{dot2tex}[mathmode]
        digraph G
        {
            rankdir=LR
            node[shape=none, label=""];
            element[label="\counter", shape="rect"];
            init -> element [headlabel="init"];
            trigger -> element [headlabel="trigger"];
            element -> ctr [taillabel="ctr"];
            element -> key_o [taillabel="key"];
        }
    \end{dot2tex}
    \caption{$\counter$}
\end{figure}

The monotonic counter primitive outputs a monotonic sequence numbers to its
output interfaces $data$. The sequence starts with the value set exactly once
by the $init$ input interfaces. Subsequent messages on the $init$ interface are
ignored. The primitive is special in that the counter value is incremented only
if a true value is received on the $trigger$ input interface.

This element can be used in conjunction with e.g. the stream comparator to
achieve freshness. If the key stream processed by the stream comparator
receives a fresh key, the comparator sends a negative trigger which makes the
counter primitive emit the same counter value as before. If the key stream
receives a key identical to the previous one, it signals a positive trigger to
the counter primitive which then increments the counter before sending it to
the $ctr$ output. Together, the counter and the key are fresh as either the
(random) key changed or the counter is increased.

\TODO{What about overflow?}

Security guarantees:

If the environment receiving the counter value has confidentiality
requirements, either the initial value or the trigger are required to be kept
confidential, as otherwise an attacker could derive the counter from its
initial value and the number of positive triggers:

\begin{equation}
    init_{in}^{C} \vee trigger_{in}^{C} \implies ctr_{out}^{C}
\end{equation}

Furthermore, \geno{} and \genm{} apply.

\subsection{Diffie-Hellman}

\subsection{$g^x$}

\begin{figure}[ht]
    \centering
    \begin{dot2tex}[mathmode]
        digraph G
        {
            rankdir=LR
            node[shape=none, label=""];
            element[label="\dhpub", shape="rect"];
            psec_i -> element [headlabel="psec"];
            element -> pub [taillabel="pub"];
        }
    \end{dot2tex}
    \caption{$\dhpub$}
\end{figure}

This primitive is the public part of a Diffie-Hellman key exchange, i.e. $g^x$
where the input interface $psec$ receives the secret value $x$ and the output
interface $pub$ sends the public value $g^x$.

Security guarantees:

By the very nature of this operation, the input value must be confidential:

\begin{equation}
    psec_{in}^{C}
\end{equation}

In addtion, the generic rules \geno{} and \genm{} apply to ensure that $psec$
is associated with the correct pub value.

\subsection{$g^{xy}$}

\begin{figure}[ht]
    \centering
    \begin{dot2tex}[mathmode]
        digraph G
        {
            rankdir=LR
            node[shape=none, label=""];
            element[label="\dhsec", shape="rect"];
            pub -> element [headlabel="pub"];
            psec -> element [headlabel="psec"];
            element -> ssec [taillabel="ssec"];
        }
    \end{dot2tex}
    \caption{$\dhpub$}
\end{figure}

This is the second part of a Diffie-Hellman exchange that calculates a shared
secret from the local secret $x$ read from the $psec$ input interface and the
$g^y$ value received from the remote party through the $pub$ input interface. 

Security guarantees:

Both, the local secret $psec$ and the resulting shared secret $ssec$ are
required to be confidential:

\begin{equation}
    psec_{in}^{C}
\end{equation}

\begin{equation}
    ssec_{out}^{C}
\end{equation}

In addition, the generic rule \genm{} applies. \TODO{Shouldn't M be required for pub/psec to prevent an attacker from making us combine invalid pub/psec values? Is that a problem anyway?}

\subsection{Cryptographic Hashes}

\begin{figure}[ht]
    \centering
    \begin{dot2tex}[mathmode]
        digraph G
        {
            rankdir=LR
            node[shape=none, label=""];
            element[label="\hash", shape="rect"];
            msg -> element [headlabel="data"];
            element -> hash [taillabel="hash"];
        }
    \end{dot2tex}
    \caption{$\hash$}
\end{figure}

This primitive represents a generic, unkeyed cryptographic hash function.

Security guarantees:

The generic rule \genm{} applies. As messages received on the input interface
$data$ do not necessarily have enough entropy, the hash primitive does not
establish confidentiality in the general case. Hence, the generic \genc{} rule
also needs to hold.

\subsection{Symmetric cryptography}

\subsubsection{Counter-mode encryption}

\begin{figure}[ht]
    \centering
    \begin{dot2tex}[mathmode]
        digraph G
        {
            rankdir=LR
            node[shape=none, label=""];
            element[label="\encctr", shape="rect"];
            pt  -> element [headlabel="pt"];
            key -> element [headlabel="key"];
            ctr -> element [headlabel="ctr"];
            element -> ct [taillabel="ct"];
        }
    \end{dot2tex}
    \caption{$\encctr$}
\end{figure}

This is a primitive implementing symmetric counter mode encryption.

Security guarantees:

The generic rule \genm{} and the following condition have to hold:

\begin{equation}
    pt_{in}^{C} \implies (key_{in}^{C} \wedge key_{in}^{M} \wedge ctr_{in}^{M})
\end{equation}

\subsubsection{Counter-mode decryption}

\begin{figure}[ht]
    \centering
    \begin{dot2tex}[mathmode]
        digraph G
        {
            rankdir=LR
            node[shape=none, label=""];
            element[label="\decctr", shape="rect"];
            ct  -> element [headlabel="ct"];
            key -> element [headlabel="key"];
            ctr -> element [headlabel="ctr"];
            element -> pt [taillabel="pt"];
        }
    \end{dot2tex}
    \caption{$\decctr$}
\end{figure}

The $\decctr$ primitive models symmetric counter mode decryption.

Security guarantees:

The generic rule \genm{} and the following condition have to hold:

\begin{equation}
    pt_{out}^{C} \implies (key_{in}^{C} \wedge key_{in}^{M} \wedge ctr_{in}^{M})
\end{equation}

\subsection{Message authentication}

\subsubsection{HMAC}

\begin{figure}[ht]
    \centering
    \begin{dot2tex}[mathmode]
        digraph G
        {
            rankdir=LR
            node[shape=none, label=""];
            element[label="\hmac", shape="rect"];
            key -> element [headlabel="key"];
            msg -> element [headlabel="msg"];
            element -> auth [taillabel="auth"];
        }
    \end{dot2tex}
    \caption{$\hmac$}
\end{figure}

The $\hmac$ element models symmetric message authentication.

Security guarantees:

The generic \genm{} rule in addition to the following rules apply:

\begin{equation}
    auth_{out}^{C} \implies key_{in}^{C} \wedge key_{in}^{M}
\end{equation}

\begin{equation}
    msg_{in}^{M} \implies key_{in}^{C} \wedge key_{in}^{M}
\end{equation}

\subsection{Digital Signatures}

\subsubsection{Signature generation}

\begin{figure}[ht]
    \centering
    \begin{dot2tex}[mathmode]
        digraph G
        {
            rankdir=LR
            node[shape=none, label=""];
            element[label="\sign", shape="rect"];
            msg_i -> element [headlabel="msg"];
            pkey -> element [headlabel="pkey"];
            skey -> element [headlabel="skey"];
            element -> msg_o [taillabel="msg"];
            element -> sig [taillabel="sig"];
        }
    \end{dot2tex}
    \caption{$\sign$}
\end{figure}

\TODO{TBD}

\subsubsection{Signature verification}

\begin{figure}[ht]
    \centering
    \begin{dot2tex}[mathmode]
        digraph G
        {
            rankdir=LR
            node[shape=none, label=""];
            element[label="\signvrfy", shape="rect"];
            msg_i -> element [headlabel="msg"];
            pkey -> element [headlabel="pkey"];
            sig -> element [headlabel="pkey"];
            element -> msg_o [taillabel="msg"];
        }
    \end{dot2tex}
    \caption{$\signvrfy$}
\end{figure}

\TODO{TBD}

\section{Security guarantees overview}

\begin{tabular}{l|c|c|c|c} \hline
Primitive   & \genc{} & \genm{} & \geno{} & Custom \\\hline
$\release$    &         & \cm     &         &        \\ 
$\const$      &         &         &         & TBD    \\
$\dhsec$      &         & \cm     &         & \cm    \\
$\encctr$     &         & \cm     &         & \cm    \\
$\decctr$     &         & \cm     &         & \cm    \\
$\hmac$       &         & \cm     &         & \cm    \\
$\comp$       &         & \cm     & \cm     &        \\
$\streamcomp$ &         & \cm     & \cm     &        \\
$\rng$        &         & \cm     & \cm     &        \\
$\counter$    &         & \cm     & \cm     & \cm    \\
$\dhpub$      &         & \cm     & \cm     & \cm    \\
$\hash$       & \cm     & \cm     &         &        \\
$\transform$  & \cm     & \cm     &         &        \\
$\permute$    & \cm     & \cm     & \cm     &        \\
$\guard$      & \cm     & \cm     & \cm     & \cm    \\
\end{tabular}

\bibliographystyle{plain}
\bibliography{../../../common/Library}

\end{document}
